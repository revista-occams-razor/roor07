\rput(2.5,-1.5){\resizebox{!}{4.5cm}{{\epsfbox{images/general/trucos1.eps}}}}
\hypertarget{trucos1}{}\label{trucos}

\pagestyle{trucos}


% -------------------------------------------------
% Cabecera
\begin{flushright}


{\color{introcolor}\mtitle{8cm}{Con un par... de l�neas}}

\msubtitle{5cm}{Los Mejores Trucos}

{\sf por Tamariz el de la Perdiz}

{\psset{linecolor=black,linestyle=dotted}\psline(-12,0)}
\end{flushright}

\vspace{2mm}
% -------------------------------------------------

\begin{multicols}{2}

\lstset{language=C,frame=tb,framesep=5pt,basicstyle=\scriptsize}


\hypertarget{mostrar-el-progreso-de-dd}{%
\sectiontext{white}{black}{\texorpdfstring{MOSTRAR EL PROGRESO DE
\texttt{dd}}{Mostrar el progreso de dd}}\label{mostrar-el-progreso-de-dd}}

La aplicaci�n \texttt{dd} nos permite copiar ficheros a bajo nivel, lo
que incluye volcar discos duros enteros en un fichero. En este caso, el
proceso puede durar bastante tiempo y por defecto, \texttt{dd} no
muestra nada en pantalla. Podemos obtener informaci�n sobre enviando la
se�al \texttt{SIGUSR1} a \texttt{dd} con un comando como:

\begin{verbatim}
$ killall -USR1 dd
\end{verbatim}

El terminal ejecutando \texttt{dd} mostrar� algo como esto:

\begin{verbatim}
$ dd if=/dev/zero bs=1 count=5M of=aa
841252+0 records in
841252+0 records out
841252 bytes (841 kB, 822 KiB) copied, \
1.31938 s, 638 kB/s
\end{verbatim}

Si queremos monitorizar peri�dicamente el estado de \texttt{dd} podemos
enviar la se�al cada cierto tiempoi (10 segundos en el ejemplo de
abajo):

\begin{verbatim}
$ watch -n10 killall -USR1 dd
\end{verbatim}

\hypertarget{mostrar-el-progreso-de-dd-de-forma-interactiva}{%
\sectiontext{white}{black}{\texorpdfstring{MOSTRAR EL PROGRESO DE \texttt{dd} DE FORMA INTERACTIVA}{Mostrar el progreso de dd de forma interactiva}}\label{mostrar-el-progreso-de-dd-de-forma-interactiva}}

Si somos m�s de barras de progreso, podemos utilizar la utilidad
\texttt{pv} que es capaz de conectar la salida est�ndar de un programa
con la entrada est�ndar de otro, contando los datos que se transfieren y
mostrando una barra de progreso.

El siguiente comando comprime y almacena en un fichero la primera
partici�n de la tarjeta SD de un SBC:

\begin{verbatim}
$ dd if=/dev/mmcblk0 | pv -s 4G -peta | \
gzip -1 > /media/sda1/sd_backup.img.gz
\end{verbatim}

Como pod�is ver, es necesario pasar el tama�o m�ximo de los datos a
transferir de forma que pueda calcular el porcentaje de los datos
transferidos. En el caso de una tarjeta SD probablemente lo sepamos. En
otros casos deberemos calcularlo.

\hypertarget{descargar-puxe1ginas-web-desde-bash}{%
\sectiontext{white}{black}{DESCARGAR P�GINAS WEB DESDE BASH}\label{descargar-puxe1ginas-web-desde-bash}}

Bash ofrece ciertas capacidades de red que nos permite enviar o recibir
datos usando TCP/IP desde nuestros scripts. Estas funciones son
raramente usadas, pero en ocasiones nos pueden resultar �til\ldots{} por
ejemplo, si en un apuro resulta que no tenemos ni \texttt{wget}, ni
\texttt{curl}, ni \texttt{lynx},\ldots. Una forma de descargar una
p�gina web o cualquier fichero usando HTTP ser�a:

\begin{verbatim}
$ exec 3<>/dev/tcp/HOST/PUERTO
$ echo -e "GET /index.html" >&3
$ cat <&3
\end{verbatim}

La primera l�nea crea un descriptor de ficheros asociado a una conexi�n
TCP al host y puerto indicado. Una vez creado podemos leer y escribir de
ese descriptor. La segunda l�nea realiza una petici�n \texttt{GET} del
fichero \texttt{index.html} y la �ltima l�nea lee el resultado del
servidor. Podemos usar esto con cualquier servicio TCP que se pueda
ejecutar de forma interactiva.

\hypertarget{usos-avanzados-de-scp}{%
\sectiontext{white}{black}{\texorpdfstring{USOS AVANZADOS DE \texttt{scp}}{Usos avanzados de scp}}\label{usos-avanzados-de-scp}}

La utilidad \texttt{scp} usa la infraestructura ofrecida por
\texttt{ssh} para copiar ficheros entre m�quinas de forma seguro y
vers�til. Estas son algunas cosas que podemos hacer con \texttt{scp}:

\begin{verbatim}
$ scp -l 400 origen destino
\end{verbatim}

Este comando limitar� el ancho de banda usado a 400 kps o si lo prefer�s
50KB/s

\begin{verbatim}
$ scp -3 origen destino
\end{verbatim}

Este comando copiar� los datos desde la m�quina de origen a la m�quina
destino usando la m�quina actual como punto intermedio. De esta forma es
posible copiar ficheros entre dos m�quinas que no se pueden comunicar
entre si, utilizando una tercera m�quina que puede conectarse con las
dos. En este modo es recomendable configurar el acceso a las m�quinas
sin contrase�a o de lo contrar�o la situaci�n puede ser un poco confusa.

\begin{verbatim}
$ scp -R origen destino
\end{verbatim}

Este comando es similar al anterior, pero en este caso estaremos
conect�ndonos a origen usando \texttt{ssh} y ejecutando \texttt{scp}
desde all�. Este modo requiere que \texttt{origen} pueda autenticarse en
el host \texttt{destino} sin utilizar contrase�a.

\end{multicols}
